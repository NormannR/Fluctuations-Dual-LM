\section{Conclusion}

In this paper, I show the importance of the contractual composition with respect to labor market flows when it comes to understand fluctuations on a dual labor market. The estimated model replicates well the main moments of a typical European dual labor market. Our approach highlights the importance of a contractual substitution effect at the hiring stage and the general-equilibrium effect it fuels by impacting the job seekers' pool. Inflation dynamics reflect the firms' hiring choices in contractual terms. As a result, employment protection legislation reforms directly impact inflation dynamics.

A few points merit further discussion, though. A potential issue in the current analysis is the implicit assumption that the labor market \emph{is} at the steady state. According to the data, temporary employment in the Euro area seems to have stabilized since the beginning of the 2000s, but the exponential rise of temporary employment between the 80s and 2000 may cast doubt on its future behavior. The absence of endogenous quitting decisions is also a limitation that needs to be highlighted, especially in the Euro Area where voluntary job-to-job transition represent a substantial share of permanent job separations and probably present a sharp pro-cyclical behavior. The exogenous character of the productivity distribution and its independence of other shocks also constitutes a problem, because this makes the model sensitive to the Lucas critique. Indeed, an employment protection legislation reform probably may modify firms' productivity distributions. With these critiques in mind, my findings should be considered as an incentive to consider the contractual composition of labor market flows in an endogenous manner when short-term horizons and monetary policy are considered.
