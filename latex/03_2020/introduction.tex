\section*{Introduction}

In the past decades, fixed-term employment has grown in importance on European labor markets. In France, for example, temporary contracts represented 5 \% of employed workers in the 80s, while it fluctuates around 15\% today. Around 80 \% of job creation occurs through temporary contracts nowadays, most of them with a stipulated duration of less than one month\footnote{See \citet{fontaine2016cdd} for an overview of the situation in European countries}. Fixed-term employment serves as a buffer against immediate workload fluctuations\footnote{See \citet{doi:10.1111/iere.12167} } and, to this extent, shares the time scale of conventional monetary policy. I estimate a New-Keynesian DSGE model with a dual labor market using Euro-area data to study the interaction between these two.

The contribution is threefold. First of all, the model is able to account for the contractual composition of hires and its fluctuations. The strong counter-cylicality of the share of temporary contracts in job creation is well rendered, among other moments characterizing a dual labor market. Secondly, the possibility to substitute temporary contracts and permanent contracts at the hiring stage plays an important role on impact. This initial substitution effect on the composition of job creation impacts the number of job seekers, which in turn influences the thickness of job creation flows. There is a direct link between fluctuations in composition of job creation and fluctuations in the quantity of job creation. Finally, I find that the newly hired workers' contractual composition and productivity intervene in inflation dynamics. Marginal and transitory reforms on employment protection legislation, which are frequent in the Euro Area\footnote{\citet{fontaine2016cdd} state that reforms are actually frequent and often marginal in Western and Southern Europe. Between 2005 and 2013, they count 17 employment protection legislation reforms in France, 49 in Italy, 38 in Spain, 23 in Greece and 17 in Portugal}, generate small and persistent movements in inflation.

The literature is scarce when it comes to consider both a truly endogenous choice between open-ended and fixed-term contracts and fluctuations in a dual labor market. The main references when it comes to study cycles with a dual labor market are the pioneering papers of \citet{sala2009flexibility}, \citet{RePEc:bde:journl:y:2010:i:04:n:04} and \citet{SJOE:SJOE1715}. These models either assume that job creation only occurs through temporary contracts, or the share of temporary contracts is constant and exogenously set. However, \citet{shimer2005cyclical} demonstrates the importance of job creation flows in the understanding of unemployment fluctuations. Moreover, given the overwhelming prominence of temporary contracts in job creation flows, the contractual composition of hires needs to be considered from a policy perspective.

In this paper, firms and workers meet as in the classic \citet{mortensen1994job} ; firms maintain vacancies while workers search for a job. When a firm and a worker meet, they draw a productivity and choose between a fixed-term contract and an open-ended contract accordingly. The resulting match then faces i.i.d shocks on its productivity. I assume that open-ended contracts are more productive than fixed-term contracts. Open-ended contracts embed a firing cost in case of endogenous separation, whereas temporary contracts stipulate an exogenous short duration without a separation cost. As a result, fixed-term contracts enable a quick and costless split in the doldrums. This productivity-flexibility trade-off is highlighted in several papers. \citet{cao2010fixed} show that highly productive workers are offered permanent contracts, because temporary workers have an incentive to search on the job, which depletes their productivity. \citet{10.2307/20485287}, as for them, introduce temporary contracts, which are less productive than permanent contracts by assumption, in firms with a decreasing-return-to-scale technology. As a result, firms hire permanent contracts until the productivity gains are offset by the expected losses from costly separations, and hire temporary contracts beyond that point. \citet{rion:halshs-02331887} is the closest paper when it comes to the economic schemes at the hiring stage.

The assumption of an existing contractual productivity wedge is fundamental and deserves justification. Temporary workers are likely to undergo successions of short employment periods and long unemployment spans \citep{fontaine2016cdd}. \citet{pissarides1992loss} shows that the latter reduce concerned  workers' skills. Moreover, fixed-term positions are mainly filled by low-skilled or unexperienced workers \citep{fontaine2016cdd}, who benefit less from on-the-job training \citep{doi:10.1111/1467-8543.00106,10.1162/154247604323068041,Albert2005,10.1093/esr/jcs011}.

The study of nominal rigidities along with fluctuations in frictional labor markets constitutes a leafy literature\footnote{See, among many others,\citet{gertler2008estimated}, \citet{trigari2009equilibrium} and \citet{christiano2016unemployment} among others}, where dualism has never been considered, as far as I know.  For the sake of comparability, I stick to \citet{thomas2009labor} when it comes to the New-Keynesian block. They introduce employment protection legislation in a Mortensen-Pissarides model along with nominal rigidities but the labor market only includes open-ended contracts.

The paper is organized as follows. Section 2 presents the model. Section 3 exposes the calibration and estimation procedure. Section 4 displays the main results of our analysis. Section 5 concludes.  